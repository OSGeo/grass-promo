\documentclass[landscape,a0,final]{a0poster}
\usepackage[dvipsnames,svgnames]{xcolor}
\usepackage{tikzposter} % here most of the things are defined
% change parameters only after this line You can also start a new column with an arbitrary 
% x-coordinate by specifying explicitly the coordinate of the new block node as follows:
\usepackage[czech]{babel}
\usepackage[utf8]{inputenc}
\usepackage{wrapfig}
\usepackage{url}

% For blibliography styling
\usepackage{natbib}

%Used for better control on code display

\usepackage[margin=\margin cm, paperwidth=197cm, paperheight=100cm]{geometry}

% \setbackgrounddarkcolor{ForestGreen!70!black}
% \setbackgroundlightcolor{YellowGreen!90!}

% \setfirstcolor{YellowGreen!80!}
% \setsecondcolor{gray!80!}
% \setthirdcolor{red!80!black}

\title{\bigskip GRASS GIS: a peer-reviewed scientific platform\\ and future research repository\bigskip}
\author{Yann Chemin$^4$, Vaclav Petráš$^3$, Anna Petrášova$^3$, Martin Landa$^2$, Soeren Gebbert$^5$, 
\\Pietro Zambelli$^6$, Markus Neteler$^1$, Peter Loewe$^7$, Magherita di Leo$^8$\\ \bigskip
$^1$ CRI, FEM, Italy, $^2$ CTU in Prague, Czech Republic, $^3$ NCSU, USA, $^4$ IWMI, Sri Lanka, 
$^5$ TICSA, Germany, $^6$ EURAC, Italy, $^7$ GNLST, Germany, $^8$ EC-JRC, Italy}

\usetemplate{1}
\setinstituteshift{1}

\setblocktitleheight{2}
\setblockspacing{1}

\begin{document}
\ClearShipoutPicture
\AddToShipoutPicture{\BackgroundPicture}
\noindent
\tikzstyle{every picture}+=[remember picture]
\begin{tikzpicture}
\initializesizeandshifts
\titleblock{90}{1}
% \setblocktitleheight{1}

\addlogo[north west]{(2,-1)}{9cm}{images/Grass_GIS}
%Please insert your institution logo here
\addlogo[north east]{(-2,-2.5)}{4cm}{images/logo_FEM_CRI}
\addlogo[north east]{(-2,-5.5)}{4cm}{images/NC_State_Seal}
\addlogo[north east]{(-8,-2.5)}{4cm}{images/Logo_cvut}
\addlogo[north east]{(-8,-6.5)}{4cm}{images/IWMI_logo}
\addlogo[north east]{(-2,-10.5)}{4cm}{images/logo_ec-jrc}

%%%%%%%%%%%%%%%%%%%%%%%%%%%%%%%%%%%%%%%%%%%%%%%%%%%%%%%%%%%%%%%%%%%%%%%%%%%%%%%%
\blocknode{Abstract}
{
\small \noindent Geographical Information Systems (GIS) is known for its capacity 
to spatially enhance the capacity of man- agement of natural resources. 
While being often used as an analytical tool, it also represents a collaborative 
scientific platform to develop new algorithms. GRASS GIS (Neteler et al., 2012 
\cite{neteler2012grass}), a free and open source GIS, is used by many scientists 
directly or through other projects such as R or QGIS to perform geoprocessing tasks. 
Thus, a large number of scientific geospatial computations depend on quality and 
correct functionality of GRASS GIS. Integrating scientific algorithms into GRASS GIS 
helps to preserve reproducibility of scientific results over time as the original author 
designed it (Rocchini \& Neteler, 2012 \cite{rocchini2012let}). Moreover, subsequent 
improvements are tracked in the source code versioning system and are immediately 
available to the public (Petras, 2014 \cite{Petras2014}). Thus, GRASS GIS acts as a 
repository of scientific peer-reviewed code and algorithm/knowledge hub for future 
generation of scientists.\vspace{5mm}\newline

With the various types of actual ET models being developed in the last 20 years, 
it becomes necessary to inter-compare methods. Most of already published ETa models 
comparisons address few number of models, and small to medium areas 
(Chemin, 2014 \cite{chemin2012distributed}; Gao and Long, 2008 \cite{gao2008intercomparison}; 
Garcia et al., 2007 \cite{garcia2007comparison}; Suleiman et al., 2008
\cite{suleiman2008intercomparison}; Timmermans et al., 2007 \cite{timmermans2007intercomparison}). 
With the large amount of remote sensing data covering the Earth, and the daily 
information available for the past ten years (i.e. Aqua/Terra-MODIS) for each pixel 
location, it becomes paramount to have a more complete comparison, 
in space and time.\vspace{5mm}\newline

To address this new experimental requirement, a distributed computing framework was 
designed, and created (Chemin, 2012 \cite{chemin2012distributed}). 
The design architecture was built from original satellite datasets to various levels 
of processing until reaching the requirement of various ETa models input dataset. 
Each input product is computed once and reused in all ETa models requiring such input. 
This permits standardization of 
inputs as much as possible to zero-in variations of models to the models 
internals/specificities. All of the ET models are available in the new GRASS GIS 
version 7 as imagery modules and replicability is complete for future 
research.\vspace{5mm}\newline

A set of modules for multiscale analysis of landscape structure was added in 1992 
by Baker et al. \cite{baker1992r}, who developed the r.le model similar to 
FRAGSTATS \cite{mcgarigal1995fragstats}, see manual. The modules were gradually 
improved to become r.li in 2006. Further development continued, with a significant 
speed up (Trac1, 2014) and new interactive user interface.\vspace{5mm}\newline
The module v.surf.rst for spatial interpolation was developed approximately 12 years 
ago, since then it was improved several times (Trac2, 2014). It is an important part 
of GRASS GIS and is even taught at geospatial modeling courses, for example 
http://courses.ncsu.edu/gis582/common/grass/interpolation\_2.html.\vspace{5mm}\newline

GRASS GIS entails several modules that constitute the result of active research on 
natural hazard. The r.sim.water simulation model (Mitas and Mitasova, 1998 \cite{Mitas1998b}) 
for overland flow under rainfall excess conditions was integrated into the Emergency 
Routing Decision Planning system as a WPS (Raghavan et al., 2014 \cite{raghavan2014deploying}). 
It was also modified by Petrasova et al., 2014 \cite{Petrasova2014} and is now part of a 
specialised software called Tangible Landscape (previously Tangible GIS), which also 
incorporated the r.damflood module.\vspace{5mm}\newline

The wildfire simulation toolset, firstly developed by Xu, 1994 \cite{xu1994simulating}, 
implementing Rothermel’s model \cite{Rothermel1983how}, available through the GRASS 
functions r.ros and r.spread, is object of active research. It has been extensively 
tested and recently adapted to European fuel types (Rodriguez-Aseretto et al.,
2013 \cite{rodriguez2013data} ; de Rigo et al., 2013 \cite{derigo2013architecture} ; 
Di Leo et al., 2013 \cite{2013_DiLeo_etAl}).
}

%%%%%%%%%%%%%%%%%%%%%%%%%%%%%%%%%%%%%%%%%%%%%%%%%%%%%%%%%%%%%%%%%%%%%%%%%%%%%%%%
%\getcurrentrow{box}
%\coordinate (funkcionalita) at (box.south west);
%\coordinate (funkcionalitaeast) at (box.east);
%\coordinate (screenshot) at (box.north west);
%
%\blocknodew[($(funkcionalita)+(20,-1)$)]{35}{References}{
\blocknode{References}{
\smallskip
\scriptsize

\begingroup
\renewcommand{\section}[2]{}%
\bibliographystyle{plain}
\bibliography{poster}
\endgroup

%\hrulefill
%\vspace{-5pt}

\begin{center}
\begin{tabular}{c}
\hspace{5mm}
\begin{minipage}{0.2\textwidth}
\includegraphics[width=0.5in]{./images/grass_qr.pdf}
\end{minipage}

\begin{minipage}{0.25\textwidth}
\small {\url{grasswiki.osgeo.org}}
\end{minipage}

\begin{minipage}{0.15\textwidth}
\includegraphics[width=0.7in]{./images/public_domain_logo}
\end{minipage}

\begin{minipage}{0.3\textwidth}
\small {\url{grass.osgeo.org}}
\end{minipage}

\begin{minipage}{0.2\textwidth}
\includegraphics[width=0.5in]{./images/grass_qr.pdf}
\end{minipage}

\end{tabular}
\end{center}

}

\startsecondcolumn

%%%%%%%%%%%%%%%%%%%%%%%%%%%%%%%%%%%%%%%%%%%%%%%%%%%%%%%%%%%%%%%%%%%%%%%%%%%%%%%%
\blocknode{}{
\bigskip

}


\startthirdcolumn
%%%%%%%%%%%%%%%%%%%%%%%%%%%%%%%%%%%%%%%%%%%%%%%%%%%%%%%%%%%%%%%%%%%%%%%%%%%%%%%
\blocknode{}{
\smallskip

}


\blocknode{Conclusions}{
\smallskip
}

\startfourthcolumn
%%%%%%%%%%%%%%%%%%%%%%%%%%%%%%%%%%%%%%%%%%%%%%%%%%%%%%%%%%%%%%%%%%%%%%%%%%%%%%%%
\blocknode{}{
\smallskip

}
%%%%%%%%%%%%%%%%%%%%%%%%%%%%%%%%%%%%%%%%%%%%%%%%%%%%%%%%%%%%%%%%%%%%%%%%%%%%%%%
\blocknode{}{

}%End of Block



\end{tikzpicture}

\end{document}
