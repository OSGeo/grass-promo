\documentclass[landscape,a0,final]{a0poster}
\usepackage[dvipsnames,svgnames]{xcolor}
\usepackage{tikzposter} % here most of the things are defined
% change parameters only after this line You can also start a new column with an arbitrary 
% x-coordinate by specifying explicitly the coordinate of the new block node as follows:
\usepackage[czech]{babel}
\usepackage[utf8]{inputenc}
\usepackage{wrapfig}
\usepackage{url}

\usepackage[margin=\margin cm, paperwidth=197cm, paperheight=100cm]{geometry}

% \setbackgrounddarkcolor{ForestGreen!70!black}
% \setbackgroundlightcolor{YellowGreen!90!}

% \setfirstcolor{YellowGreen!80!}
% \setsecondcolor{gray!80!}
% \setthirdcolor{red!80!black}

\title{\bigskip GRASS GIS Vector State of the Art  -  Gearing towards GRASS GIS 7 \bigskip}
\author{Markus Metz$^1$, Martin Landa$^2$, Anna Petrasova$^3$, Vaclav Petras$^3$, Yann Chemin$^4$, Markus Neteler$^1$ and The GRASS GIS Development Team\\ \bigskip
$^1$ CRI, FEM, Italy, $^2$ CTU, Czech Republic, $^3$ NCSU, USA, $^4$ IWMI, Sri Lanka}

\usetemplate{1}
\setinstituteshift{1}

\setblocktitleheight{2}
\setblockspacing{1}

\begin{document}
\ClearShipoutPicture
\AddToShipoutPicture{\BackgroundPicture}
\noindent
\tikzstyle{every picture}+=[remember picture]
\begin{tikzpicture}
\initializesizeandshifts
\titleblock{123.8}{1}
% \setblocktitleheight{1}

\addlogo[north west]{(2,-1)}{9cm}{./svg_images/Grass_GIS.pdf}
%Please insert your institution logo here
\addlogo[north east]{(-2,-2.5)}{4cm}{./svg_images/IWMI_logo.pdf}
\addlogo[north east]{(-2,-6.5)}{4cm}{./svg_images/IWMI_logo.pdf}
\addlogo[north east]{(-8,-2.5)}{4cm}{./svg_images/IWMI_logo.pdf}
\addlogo[north east]{(-8,-6.5)}{4cm}{./svg_images/IWMI_logo.pdf}

%%%%%%%%%%%%%%%%%%%%%%%%%%%%%%%%%%%%%%%%%%%%%%%%%%%%%%%%%%%%%%%%%%%%%%%%%%%%%%%%
\blocknode{Abstract}{
\small GRASS GIS, commonly referred to as GRASS (Geographic Resources Analysis Support System),
is a free Geographic Information System (GIS) software used for geospatial data management and analysis,
image processing, graphics/maps production, spatial modeling, and visualization [1].

GRASS GIS 7 started its development by the branching out of GRASS GIS 6.x from the main trunk of code (rev 31142).
This was done on 27th of April 2008, and a large amount of changes took place since that date, more 
are still underway.
\begin{itemize}
 \item Raster library in GRASS 7 (ongoing)
 \item Vector library in GRASS 7 (ongoing)
 \item Raster3D (volume) library and modules in GRASS 7 (finished)
 \item Temporal extension for GRASS 7 (finished) 
\end{itemize}

GRASS GIS' capacity in remote sensing has also been greatly improved with additions for version 7.
}

%%%%%%%%%%%%%%%%%%%%%%%%%%%%%%%%%%%%%%%%%%%%%%%%%%%%%%%%%%%%%%%%%%%%%%%%%%%%%%%%
\blocknode{Linear features extraction}{
\small
blabla
\begin{center}
 %\includegraphics[width=0.48\textwidth]{./images/imagery_spot_original}
 %\hspace{10mm}
 %\includegraphics[width=0.48\textwidth]{./images/imagery_spot_edge_1}
 %\newline
 Figure 1: Canny edge detector on a road network [2]
\end{center}
}


%%%%%%%%%%%%%%%%%%%%%%%%%%%%%%%%%%%%%%%%%%%%%%%%%%%%%%%%%%%%%%%%%%%%%%%%%%%%%%%%
\getcurrentrow{box}
\coordinate (funkcionalita) at (box.south west);
\coordinate (funkcionalitaeast) at (box.east);
\coordinate (screenshot) at (box.north west);

\blocknodew[($(funkcionalita)+(20,-1)$)]{35}{References}{
\scriptsize
\begin{center}
\begin{tabular}{rp{0.9\textwidth}}
[1] & Neteler \& Bowman \&  Landa \& Metz, 2012. Environment \& Modeling Software, 31:124-130\\{}
[2] & Petráš, 2012. M.Sc. Thesis, OSGeoREL, FCE CTU, Prague.\\{}
[3] & Kratochvílová \& Petráš, 2013. OSGeoREL, FCE CTU, Prague.\\{}
[4] & Neteler \& Grasso \& Michelazzi \& Miori \& Merler \& Furlanello, 2005. 
International Journal of Geoinformatics, 1(1): 51-61.
\end{tabular}
\end{center}
\smallskip
\hrulefill
\vspace{-5pt}

\begin{center}
\begin{tabular}{cp{0.9\textwidth}}
\begin{minipage}{0.15\textwidth}
\includegraphics[width=0.7in]{./images/iwmi_qr.pdf}
\end{minipage}

\begin{minipage}{0.3\textwidth}
\small {\url{www.iwmi.org}}
\end{minipage}

\begin{minipage}{0.15\textwidth}
\includegraphics[width=0.7in]{./images/grass_qr.pdf}
\end{minipage}

\begin{minipage}{0.3\textwidth}
\small {\url{grass.osgeo.org}}
\end{minipage}
\end{tabular}
\end{center}

\hrulefill
\vspace{14pt}
\begin{center}
\newcommand{\logowidth}{5em}
\newcommand{\logospace}{\hspace{0.1em}}
\noindent
\includegraphics[width=\logowidth]{./svg_images/public_domain_logo.pdf}
\raisebox{0.7\height}{\logospace 2014 GRASS Development Team}
\end{center}
}

\startsecondcolumn

%%%%%%%%%%%%%%%%%%%%%%%%%%%%%%%%%%%%%%%%%%%%%%%%%%%%%%%%%%%%%%%%%%%%%%%%%%%%%%%%
\blocknode{Interactive supervised classification}{
This
\begin{center}
 %\includegraphics[width=0.47\textwidth]{./images/wxiclass1.png}
 %\newline
 Figure 3: Interactive image classification with coincidence plots (left side) \& histograms (right side)
\end{center}
}

%%%%%%%%%%%%%%%%%%%%%%%%%%%%%%%%%%%%%%%%%%%%%%%%%%%%%%%%%%%%%%%%%%%%%%%%%%%%%%%%
\blocknode{Satellite imagery products}{
blabla
}

\startthirdcolumn
%%%%%%%%%%%%%%%%%%%%%%%%%%%%%%%%%%%%%%%%%%%%%%%%%%%%%%%%%%%%%%%%%%%%%%%%%%%%%%%%
\blocknode{Lidar}{
\smallskip
The Lidar library ({\url {www.liblas.org}}) included in GRASS GIS permits the import of Lidar (.las)
data in raster (r.in.lidar using statistics of choice) or in vector format (v.in.lidar). 
Author Markus Metz tested r.in.lidar with a 705Gb .las file. \newline
On-farm water storage study with lidar data in NSW (Australia) developed a full remote sensing monitoring methodology
of water availability with lidar-based bathymetric survey and multi-source remote sensing survey [8].\newline
\begin{center}
 %\includegraphics[width=0.4\textwidth]{./images/ofs1}
 %\newline
 Figure 5: On-Farm-Water-Storage Lidar survey and Depth-Volume-Area surveying [8]
\end{center}
}

\blocknode{Other Improvements \& Additions}{
\smallskip

{\bf Remanufacturing, performance improvement}


{\bf Other functions}

\begin{itemize}
 \item {\bf v.} blabla
 \item {\bf v.} blabla
 \item {\bf v.} blabla
 \item {\bf v.} blabla
 \item {\bf v.} blabla
\end{itemize}
}

\startfourthcolumn
%%%%%%%%%%%%%%%%%%%%%%%%%%%%%%%%%%%%%%%%%%%%%%%%%%%%%%%%%%%%%%%%%%%%%%%%%%%%%%%%
\blocknode{Multi- and hyperspectral data analysis}{
\smallskip
Unmixing 

\begin{center}
 %\includegraphics[width=0.4\textwidth]{./images/unmix_pixels_spectrum.pdf}
 %\newline
 Figure 7: Unmixing principle (left), end-members selection (right), error space (below)
\end{center}
}

\end{tikzpicture}

\end{document}
