% License: CC BY-SA
% Authors: See the authors below and see also acknowledgement for authors of some images or research

\documentclass[25pt, margin=0mm, innermargin=25mm, blockverticalspace=25mm, colspace=25mm, subcolspace=8mm]{tikzposter}
\geometry{paperwidth=63in,paperheight=42in}

% to stretch boxes over whole paper with custor paper size
\makeatletter
\setlength{\TP@visibletextwidth}{\textwidth-2\TP@innermargin}
\setlength{\TP@visibletextheight}{\textheight-2\TP@innermargin}
\makeatother

% Fira Sans, GRASS GIS branding sans serif font
\usepackage{FiraSans}
\renewcommand*\oldstylenums[1]{{\firaoldstyle #1}}

% EB Garamond, GRASS GIS branding serif font
% note that EB Garamond does not have bold
\usepackage[cmintegrals,cmbraces]{newtxmath}
\usepackage{ebgaramond-maths}

% might be needed for both font packages
\usepackage[T1]{fontenc}

% uncomment for all sans serif
% \renewcommand{\familydefault}{\sfdefault}


\usepackage[utf8]{inputenc}
\usepackage{wrapfig}
\usepackage[hidelinks]{hyperref}

% For bibliography styling
%% TODO: all names should be abbreviated
\usepackage{natbib}

\definecolor{textcolor}{HTML}{000000}

\definecolor{titleTextColor}{HTML}{000000}
\definecolorpalette{grassColorPalette} {
  \definecolor{colorOne}{HTML}{419041}
  % \definecolor{colorTwo}{HTML}{cccccc}
  \definecolor{colorTwo}{HTML}{dddddd}
  \definecolor{colorThree}{HTML}{F1B52D}
  % \definecolor{colorThree}{HTML}{EFA126}
}

\usetheme{Default}
\usetitlestyle{Empty}
\usecolorstyle[colorPalette=grassColorPalette]{Britain}
\colorlet{backgroundcolor}{white}

\title{
\Huge
\textcolor{titleTextColor}{
\textsf{
% \textbf{
\fontsize{130}{100}\selectfont
\textbf{GRASS}\,{\firalight GIS}: A General-purpose Geospatial Research Tool
% }
}
}
}

\newlength{\grasslogoheight}
\setlength{\grasslogoheight}{0.09\textheight}
\newlength{\instlogoheight}
\setlength{\instlogoheight}{0.33\grasslogoheight}

% \setlength{\blocktitleheight}{0.02\textheight}

% style for institute numbers
\newcommand{\inst}[1]{\hspace{2pt}$^{\mbox{\normalsize#1}}$\hspace{-7pt}}
\newcommand{\instlist}[1]{\hspace{1pt}$^{\mbox{\normalsize#1}}$\hspace{2pt}}

\author{
Helena Mitasova\inst{1},
Vaclav Petras\inst{1}\,*\hspace{-7pt},
Anna Petrasova\inst{1},
\&
Markus Neteler\inst{2}
\\
% AGU allows to specify a scientific team, but it does not seem to fit
% with our case 100%; simply using formatting which is in the AGU program
% (but adding note)
Scientific Team: GRASS GIS Development Team**
}
\institute{
\large
\instlist{1}Center for Geospatial Analytics, North Carolina State University, USA;
\instlist{2}mundialis GmbH \& Co. KG, Germany;
*Corresponding author: wenzeslaus@gmail.com, vpetras@ncsu.edu;
% using Black Duck Open Hub to ``guestimate'' that
**Over 10 other members of the core team and numerous other contributors
\\[1.7cm]
\includegraphics[height=3cm]{ncstate}%
\hspace{1cm}%
\includegraphics[height=3cm]{cga}%
\hspace{3cm}%
\includegraphics[height=3.4cm]{mundialis}%
}

\hypersetup
{
    pdfauthor={H. Mitasova, V. Petras, A. Petrasova, M. Neteler},
    pdfsubject={AGU Fall Meeting 2018 Poster},
    pdftitle={GRASS GIS: A General-purpose Geospatial Research Tool},
    pdfkeywords={GIS, algorithms, methods, preservation, science, reproducibility}
}

% \usetemplate{1}
% \setinstituteshift{1}

% \setblocktitleheight{2}
% \setblockspacing{1}

\graphicspath{{images/}{logos/}}

\newcommand{\blocktitlewrap}[1]{\textnormal{\textsf{\textsc{\huge#1}}}}
% it is not possible (?) to change block title in the class, using wrapper
% the command introduced using:
%   sed -i 's/\\block{\([^}]*\)}/\\block{\\blocktitlewrap{\1}}/g' main.tex

% bullet point style
\renewcommand{\labelitemi}{\textcolor{gray}{$\bullet$}\hspace{0.5ex}}

% GRASS module
\newcommand{\gmodule}[1]{\href{http://grass.osgeo.org/grass72/manuals/#1.html}{\emph{#1}}}
\newcommand{\gamodule}[1]{\href{http://grass.osgeo.org/grass72/manuals/addons/#1.html}{\emph{#1}}}
\newcommand{\gmodulenolink}[1]{\emph{#1}}

\begin{document}

\node[above left,opacity=0.99,inner sep=0pt,outer sep=4cm] at (bottomleft -| topright)%
  {\includegraphics[width=0.3\paperwidth]{grass}};

\maketitle[width=0.92\textwidth]

% \maketitle
% \addlogo[north west]{(2,-1)}{9cm}{images/Grass_GIS}
%Please insert your institution logo here
% \addlogo[north east]{(-2,-2.5)}{4cm}{images/logo_FEM_CRI}
% \addlogo[north east]{(-2,-5.5)}{4cm}{images/NC_State_Seal}
% \addlogo[north east]{(-8,-2.5)}{4cm}{images/Logo_cvut}
% \addlogo[north east]{(-8,-6.5)}{4cm}{images/IWMI_logo}
% \addlogo[north east]{(-2,-10.5)}{4cm}{images/logo_ec-jrc}

\begin{columns}

% Abstract
%
% GRASS GIS (grass.osgeo.org) is an open source software for geospatial analysis,
% remote sensing, general geoprocessing and visualization. It is a
% community-driven project with over 35 years of continuous software development
% based on scientific expertise from many geospatial fields. It is characterized
% by long-term releases, stable APIs, and emphasis on science. GRASS GIS
% distribution strives to provide single integrated environment for 2D and 3D
% raster analysis, image processing, vector data analysis, and spatio-temporal
% data processing. It supports large raster files (billions of cells), vector
% topology, coupling with databases, and 64 bit memory. New code based on recent
% research is typically contributed to GRASS GIS Addons repository. Mature, widely
% used code is then moved to the main code base to maximize integration and
% availability. Whether it is addons or the main code base, code is usually
% maintained by the community and preserved in long term even in cases when the
% original author no longer supports the code. To support the needs of scientists,
% the documentation includes not only links to the source code, its history, and
% its authors, but also links to research papers that describe the algorithms
% implemented in the modules.
%
% Code is not only maintained but also extended and improved. For example,
% watershed and stream extraction using least cost path approach was implemented
% in 1989 and extended for massive datasets in 2011. Similarly, vector topology
% cleaning was introduced in 2002, updated over time and substantially improved in
% 2016. Another example is a solar energy module available since 1993 and
% parallelized in 2017. The current stable version of GRASS GIS 7.4 provides new
% features for geosciences such as temporal framework with temporal algebra for
% large time-series processing, fusion of elevation models from various sensors,
% 3D flows, landform detection, image segmentation, simplified batch processing,
% and integrated Python editor among others. GRASS GIS runs in various
% environments including Linux, Mac, Windows, Docker, Raspberry Pi, and on HPC
% clusters. The modules are written in Python, C, or C++. Besides command line
% interface, GRASS GIS provides a graphical user interface, a Python API, and a C
% API. Interfaces with other languages such as R and Ruby are supplied by
% collaborating communities.

%%%%%%%%%%%%%%%%%%%%%%%%%%%%%%%%%%%%%%%%%%%%%%%%%%%%%%%%%%%%%%%%%%%%%
%%%%%%%%%%%%%%%%%%%%%%%%%%%%%%%%%%%%%%%%%%%%%%%%%%%%%%%%%%%%%%%%%%%%%
%%%%%%%%%%%%%%%%%%%%%%%%%%%%%%%%%%%%%%%%%%%%%%%%%%%%%%%%%%%%%%%%%%%%%
%%%%%%%%%%%%%%%%%%%%%%%%%%%%%%%%%%%%%%%%%%%%%%%%%%%%%%%%%%%%%%%%%%%%%
\column{0.25}

%%%%%%%%%%%%%%%%%%%%%%%%%%%%%%%%%%%%%%%%%%%%%%%%%%%%%%%%%%%%%%%%%%%%%%%%%%%%%%%
\block{\blocktitlewrap{Overview}}{

\Large


\begin{itemize}
% Basic Topics
 \item Community-driven project
 \item 35 years of continuous software development
 \item long-term releases, stable APIs, and emphasis on science
 \item single integrated environment for 2D and 3D raster analysis, image processing, vector data analysis, and spatio-temporal data processing
\end{itemize}

}

%%%%%%%%%%%%%%%%%%%%%%%%%%%%%%%%%%%%%%%%%%%%%%%%%%%%%%%%%%%%%%%%%%%%%%%%%%%%%%%
\block{\blocktitlewrap{Functionality}}{

\Large

GRASS GIS
distribution strives to provide single integrated environment for 2D and 3D
raster analysis, image processing, vector data analysis, and spatio-temporal
data processing. It supports large raster files (billions of cells), vector
topology, coupling with databases, and 64 bit memory.

}

%%%%%%%%%%%%%%%%%%%%%%%%%%%%%%%%%%%%%%%%%%%%%%%%%%%%%%%%%%%%%%%%%%%%%%%%%%%%%%%
\block{\blocktitlewrap{Code Life Cycle}}{

\Large

New code based on recent
research is typically contributed to GRASS GIS Addons repository. Mature, widely
used code is then moved to the main code base to maximize integration and
availability. Whether it is addons or the main code base, code is usually
maintained by the community and preserved in long term even in cases when the
original author no longer supports the code. To support the needs of scientists,
the documentation includes not only links to the source code, its history, and
its authors, but also links to research papers that describe the algorithms
implemented in the modules.

}

%%%%%%%%%%%%%%%%%%%%%%%%%%%%%%%%%%%%%%%%%%%%%%%%%%%%%%%%%%%%%%%%%%%%%
\block{\blocktitlewrap{Example: Spatio-Temporal Data Analysis}}{
The time dimension was introduced in GRASS GIS 7.0 for vector maps, rasters, and 3D rasters
which transformed GRASS GIS into a fully-featured temporal GIS \citep{Gebbert20141, gebbert2015grass}.
Time series of map layers are managed in space time datasets, a new data type in GRASS GIS,
and are still accessible also as individual map layers.
Based on the GRASS GIS Temporal Framework Python application programming interface (API),
more than 50 modules were implemented to manage, analyze, process
and visualize space time datasets.
More than 100,000 map layers can be now handled efficiently in GRASS GIS.
%
Example usages of this functionality include
analysis of the
European Climate Assessment \& Dataset ECA\&D \citep{Haylock2008_climate_series}
and temperate climate zone in the European Union identification \citep{Gebbert20141}.

The new temporal modules (staring with \gmodulenolink{t.})
work beside well established \gmodule{r.series} module
and specialized modules such as \gamodule{r.hants} implemented according to \cite{roerink2000reconstructing}
or \gamodule{r.seasons}.
Latest addition includes raster and vector temporal algebra
which can be used for tasks such as computing hydrothermal coefficient for a time series of climate data
using the actual mathematical formula \citep{leppelt2015grass}.

\vspace*{1.5cm}

\begin{minipage}{\linewidth}
\centering
\includegraphics[width=.7\linewidth]{images/temporal_precip_temp}
\\
Creating a synchronized animation of monthly total precipitation and mean temperature for NC, USA
\end{minipage}

\vspace*{1cm}

}



%%%%%%%%%%%%%%%%%%%%%%%%%%%%%%%%%%%%%%%%%%%%%%%%%%%%%%%%%%%%%%%%%%%%%
%%%%%%%%%%%%%%%%%%%%%%%%%%%%%%%%%%%%%%%%%%%%%%%%%%%%%%%%%%%%%%%%%%%%%
%%%%%%%%%%%%%%%%%%%%%%%%%%%%%%%%%%%%%%%%%%%%%%%%%%%%%%%%%%%%%%%%%%%%%
%%%%%%%%%%%%%%%%%%%%%%%%%%%%%%%%%%%%%%%%%%%%%%%%%%%%%%%%%%%%%%%%%%%%%
\column{0.25}


%%%%%%%%%%%%%%%%%%%%%%%%%%%%%%%%%%%%%%%%%%%%%%%%%%%%%%%%%%%%%%%%%%%%%
%%%%%%%%%%%%%%%%%%%%%%%%%%%%%%%%%%%%%%%%%%%%%%%%%%%%%%%%%%%%%%%%%%%%%%%%%%%%%%%%%
\block{\blocktitlewrap{Example: Water, Floods and Erosion}}{

GRASS GIS entails several modules that constitute the result of active research on natural hazards.
The \gmodule{r.sim.water} simulation model \citep{Mitas1998b}
for overland flow with spatially variable rainfall excess conditions was integrated into the Emergency
Routing Decision Planning system as a WPS \citep{raghavan2014deploying}.
The module \gmodule{r.sim.water} together with
the module \gmodule{r.sim.sediment} for erosion-deposition modeling
implements a path sampling algorithm which is robust and easy to parallelize.
A unique least cost path algorithm which doesn't need any depression filling
was implemented in \gmodule{r.watershed} module in 1989
and it was updated for massive data sets in 2011.
% TODO: refs
The \gmodule{r.sim.water} module was also utilized by \cite{Petrasova2014} and is now part of
\emph{Tangible Landscape}, a tangible GIS system, which also incorporated \gmodule{r.damflood},
a dam break inundation simulation by \cite{cannata2012two}.

\bigskip

\vspace*{1cm}

\centering

\begin{minipage}{0.49\linewidth}
\centering
\includegraphics[width=0.7\textwidth]{rsimwater_architects}
\end{minipage}
~
\begin{minipage}{0.49\linewidth}
\includegraphics[width=\textwidth]{damflood_tangible}
\end{minipage}

\bigskip

\begin{minipage}{0.49\linewidth}
\centering
Overland flow simulated by \gmodule{r.sim.water} used for landscape
architecture design in Tangible Landscape environment
(Historical master plan for Lake Raleigh, NC, USA)
\end{minipage}
~
\begin{minipage}{0.49\linewidth}
Dam breach on Lake Raleigh (NC, USA) in Tangible Landscape environment simulated using \gamodule{r.damflood} module
\end{minipage}


% \vspace*{1.4cm}
}

%%%%%%%%%%%%%%%%%%%%%%%%%%%%%%%%%%%%%%%%%%%%%%%%%%%%%%%%%%%%%%%%%%%%%%%%%%%%%%%%
\block{\blocktitlewrap{Example: Vector Network Analysis}}{
In 2003 Radim Blazek introduced shortest path analysis (\gmodule{v.net.path}),
traveling salesman (\gmodule{v.net.salesman}),
and several other modules for vector network analysis (staring with \gmodule{v.net}).
Over the years, the number of available algorithms increased to over 15 modules
including, for example, centrality measures and connected components added in 2009 by Daniel Bundala.
In 2014 Stepan Turek implemented turns support into all relevant vector network modules.
Also since GRASS GIS 7.0 all vector network analysis tools provide fine control over node costs.
% combination with r.cost/r.walk workflows \citep{Petrasova2014}

\bigskip

\centering
\begin{minipage}{0.9\linewidth}
\begin{center}
\includegraphics[width=\textwidth]{network_trips}
A series of trips to collect samples across North Carolina, USA
for water quality measurements
as part of the Polyfluorinated Alkyl Substance Testing (PFAST) project.
Trips are distinguished by color and were computed using multiple
iterations of traveling salesman (\gmodule{v.net.salesman}).
\end{center}
\end{minipage}

}

%%%%%%%%%%%%%%%%%%%%%%%%%%%%%%%%%%%%%%%%%%%%%%%%%%%%%%%%%%%%%%%%%%%%%
%%%%%%%%%%%%%%%%%%%%%%%%%%%%%%%%%%%%%%%%%%%%%%%%%%%%%%%%%%%%%%%%%%%%%
%%%%%%%%%%%%%%%%%%%%%%%%%%%%%%%%%%%%%%%%%%%%%%%%%%%%%%%%%%%%%%%%%%%%%
%%%%%%%%%%%%%%%%%%%%%%%%%%%%%%%%%%%%%%%%%%%%%%%%%%%%%%%%%%%%%%%%%%%%%
\column{0.25}

%%%%%%%%%%%%%%%%%%%%%%%%%%%%%%%%%%%%%%%%%%%%%%%%%%%%%%%%%%%%%%%%%%%%%%%%%%%%%%%
\block{\blocktitlewrap{Example: Image Segmentations}}{
\gamodule{r.smooth.seg} (formerly \gmodulenolink{r.seg})
was created by Alfonso Vitti \citep{vitti2008free, vitti2012mumford}
and implemented a piece-wise smooth approximation of the original image
according to \cite{mumford1989optimal} and \cite{march1997variational}
which can be used to reduce noise in the original image.

This supplemented \gmodule{r.clump} available from 1980s
which groups pixels with the same categories (or integer values).
The latest version of \gmodule{r.clump} coming in GRASS GIS 7.4
supports multiple image bands (or any rasters maps) as input.
Clumping of cells based on threshold value and clumping
of double precision floating point input is supported as well in the new version.

Eric Momsen implemented initial version of region-growing image segmentation in 2012
and Markus Metz then extended and optimized the code resulting in
the inclusion of the \gmodule{i.segment} module into GRASS GIS 7.0
and final replacement of old multi-resolution classification module of the same name
which was available in 1992 \citep{zhuang1992image}.
Another module called \gamodule{i.segment.hierarchical} by Pietro Zambelli is based on \gmodule{i.segment}
and performs parallelized hierarchical segmentation.

\vspace*{0.7cm}

\begin{minipage}{0.5\linewidth}
\begin{center}
\includegraphics[width=\textwidth]{superpixels_slic_pseudo}
\end{center}
\end{minipage}
~
\begin{minipage}{0.5\linewidth}
\begin{center}
\includegraphics[width=\textwidth]{superpixels_slic_colored}
\end{center}
\end{minipage}
\vspace{2mm}
\begin{center}
Superpixels (black outlines) on pseudo-color image of central Wake county, NC, USA (left)
and the same superpixels colored according to the mean NDVI value per pixel (right).
\end{center}

\vspace*{1cm}

In 2016 GRASS GIS implementation of SLIC Superpixels segmentation \citep{achanta2010epfl, achanta2012slic}
was requested by the community
and Rashad Kanavath and Markus Metz implemented \gmodule{i.superpixels.slic}
which provides users both with SLIC and SLIC0 methods.


}

%%%%%%%%%%%%%%%%%%%%%%%%%%%%%%%%%%%%%%%%%%%%%%%%%%%%%%%%%%%%%%%%%%%%%%%%%%%%%%%%
\block{\blocktitlewrap{References}}{

\vspace{-0.2cm}
\scriptsize

% \newcommand{\blocksectiontitle}[1]{\subsubsection*{\textcolor{gray}{\textsf{#1}}}}
\newcommand{\blocksectiontitle}[1]{\textbf{#1}}

%\blocksectiontitle{References}
\begingroup
\renewcommand{\section}[2]{}%
\bibliographystyle{apalike}
\bibliography{poster}
\endgroup

}

%%%%%%%%%%%%%%%%%%%%%%%%%%%%%%%%%%%%%%%%%%%%%%%%%%%%%%%%%%%%%%%%%%%%%
%%%%%%%%%%%%%%%%%%%%%%%%%%%%%%%%%%%%%%%%%%%%%%%%%%%%%%%%%%%%%%%%%%%%%
%%%%%%%%%%%%%%%%%%%%%%%%%%%%%%%%%%%%%%%%%%%%%%%%%%%%%%%%%%%%%%%%%%%%%
%%%%%%%%%%%%%%%%%%%%%%%%%%%%%%%%%%%%%%%%%%%%%%%%%%%%%%%%%%%%%%%%%%%%%
\column{0.25}

%%%%%%%%%%%%%%%%%%%%%%%%%%%%%%%%%%%%%%%%%%%%%%%%%%%%%%%%%%%%%%%%%%%%%
\block{\blocktitlewrap{Example: Lidar Data Processing}}{

Filtering of ground and non-ground points was included into GRASS GIS
early on in the lidar era in the \gmodule{v.lidar.edgedetection} group of modules.
% TODO: ref
% v.lidar.correction Corrects the v.lidar.growing output. It is the last of the three algorithms for LIDAR .
% v.lidar.edgedetection Detects the object's edges from a LIDAR data set.
% v.lidar.growing Building contour determination and Region Growing algorithm for determining the building inside
The module \gmodule{v.surf.rst} for spatial interpolation was developed approximately 20 years
ago, but since then it has been improved several times \citep{tracvsurfrst}
including recent parallelization which will be included in GRASS GIS 7.4.
Currently the module is used for both digital terrain model interpolation and interpolations in general.

The module \gmodule{r.in.lidar} statistically analyzes massive point clouds.
% TODO: base_raster, ref
For advanced general point analysis GRASS GIS
\gmodule{v.outlier} implemented originally specifically for lidar data
and recently also \gmodule{v.cluster}
implementing variety of clustering methods including
DBSCAN (Density-Based Spatial Clustering of Applications with Noise)
and OPTICS (Ordering Points to Identify the Clustering Structure).
The \gmodule{v.outlier} module serves as a base for a user contributed module \gamodule{v.lidar.mcc}
implementing Multiscale Curvature Classification (MCC).
% TODO: ref

% v.delaunay v.voronoi New option to create Voronoi diagrams for areas.

\centering
\begin{minipage}{0.48\linewidth}
\centering
\includegraphics[width=0.7\textwidth]{elevation_lidar}
\end{minipage}
~
\begin{minipage}{0.48\linewidth}
\centering
\includegraphics[width=\textwidth]{lidar_profile}
\end{minipage}

\bigskip

\begin{minipage}{0.48\linewidth}
\centering
Digital elevation model interpolated from lidar point clouds
using \gmodule{v.surf.rst}. Data are showing tillage in an agricultural field near Raleigh (North Carolina, USA)
\end{minipage}
~
\begin{minipage}{0.48\linewidth}
\centering
Profile (vertical slice) of a small portion of a point cloud showing tree structure as captured by the returns
(\gamodule{v.profile.points})
\end{minipage}

}



\end{columns}

\node[above left,opacity=0.5,inner sep=0pt,outer sep=4cm] at (bottomleft -| topright)%
  {\includegraphics[width=0.3\paperwidth]{grass}};

\end{document}
