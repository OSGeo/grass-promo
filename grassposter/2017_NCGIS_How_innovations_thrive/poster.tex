% License: CC BY-SA
% Authors: See authors below and see also acknowledgement for authors of some images or research

\documentclass[25pt, margin=0mm, innermargin=15mm, blockverticalspace=15mm, colspace=15mm, subcolspace=8mm]{tikzposter}
\geometry{paperwidth=82in,paperheight=41in}

% to stretch boxes over whole paper with custor paper size
\makeatletter
\setlength{\TP@visibletextwidth}{\textwidth-2\TP@innermargin}
\setlength{\TP@visibletextheight}{\textheight-2\TP@innermargin}
\makeatother


\usepackage[utf8]{inputenc}
\usepackage{wrapfig}
\usepackage[hidelinks]{hyperref}

% For bibliography styling
%% TODO: all names should be abbreviated
\usepackage{natbib}


\definecolor{textcolor}{HTML}{000000}

\definecolor{titleTextColor}{HTML}{009000}
\definecolorpalette{grassColorPalette} {
  \definecolor{colorOne}{HTML}{419041}
  % \definecolor{colorTwo}{HTML}{cccccc}
  \definecolor{colorTwo}{HTML}{dddddd}
  \definecolor{colorThree}{HTML}{F1B52D}
  % \definecolor{colorThree}{HTML}{EFA126}
}

\usetheme{Rays}
\usecolorstyle[colorPalette=grassColorPalette]{Britain}

\title{
\Huge
\textcolor{titleTextColor}{
\textsf{\textbf{
\fontsize{85}{60}\selectfont
How innovations thrive in GRASS GIS
}}
}
}

\newlength{\grasslogoheight}
\setlength{\grasslogoheight}{0.09\textheight}
\newlength{\instlogoheight}
\setlength{\instlogoheight}{0.33\grasslogoheight}

\titlegraphic{
\begin{minipage}{0.3\linewidth}
\includegraphics[height=\grasslogoheight]{grass}
\hspace{2em}
\includegraphics[height=\grasslogoheight]{osgeo_project}
\end{minipage}
\hfill
\begin{minipage}{0.20\linewidth}
\setlength{\baselineskip}{120pt}
\begin{flushright}
\includegraphics[height=\instlogoheight]{ncstate}
~
\includegraphics[height=\instlogoheight]{ctu_prague}
~
\includegraphics[height=\instlogoheight]{dwd}
~
\includegraphics[height=\instlogoheight]{mundialis}
~
\includegraphics[height=\instlogoheight]{fem_cri}
~
\includegraphics[height=\instlogoheight]{iwmi}
~
\includegraphics[height=\instlogoheight]{ec_jrc}
~
\includegraphics[height=\instlogoheight]{eurac}
\end{flushright}
\end{minipage}
\vspace{-\grasslogoheight}
}

% \setlength{\blocktitleheight}{0.02\textheight}

% style for institute numbers
\newcommand{\inst}[1]{\hspace{2pt}$^{\mbox{\normalsize#1}}$\hspace{-7pt}}
\newcommand{\instlist}[1]{\hspace{1pt}$^{\mbox{\normalsize#1}}$\hspace{2pt}}

\author{
V\'{a}clav Petr\'{a}\v{s}\inst{1},
Yann Chemin\inst{2},
Martin Landa\inst{3},
Thomas Leppelt\inst{4},
Pietro Zambelli\inst{5},
Luca Delucchi\inst{6},
Margherita Di Leo\inst{7},
S\"{o}ren Gebbert,
and
Markus Neteler\inst{8}
}
\institute{
\large
\instlist{1}North Carolina State University, USA (wenzeslaus@gmail.com, vpetras@ncsu.edu);
\instlist{2}IWMI, Sri Lanka;
\instlist{3}FCE CTU in Prague, Czech Republic;
\instlist{4}DWD, Germany;\\
\instlist{5}EURAC Research, Institute for Renewable Energy, Italy;
\instlist{6}Fondazione Edmund Mach, Research and Innovation Centre, Italy;
\instlist{7}European Commission, JRC, Italy;
\instlist{8}mundialis GmbH \& Co. KG, Germany;
}

\hypersetup
{
    pdfauthor={V. Petras, Y. Chemin, M. Landa, S. Gebbert, P. Zambelli, M. Neteler, M. Di Leo},
    pdfsubject={},
    pdftitle={How innovations thrive in GRASS GIS},
    pdfkeywords={GIS, algorithms, methods, preservation, science, reproducibility}
}

% \usetemplate{1}
% \setinstituteshift{1}

% \setblocktitleheight{2}
% \setblockspacing{1}

\graphicspath{{images/}{logos/}}

\newcommand{\blocktitlewrap}[1]{\textsf{\textbf{\huge#1}}}
% it is not possible (?) to change block title in the class, using wrapper
% the command introduced using:
%   sed -i 's/\\block{\([^}]*\)}/\\block{\\blocktitlewrap{\1}}/g' main.tex

% GRASS module
\newcommand{\gmodule}[1]{\href{http://grass.osgeo.org/grass72/manuals/#1.html}{\emph{#1}}}
\newcommand{\gamodule}[1]{\href{http://grass.osgeo.org/grass72/manuals/addons/#1.html}{\emph{#1}}}
\newcommand{\gmodulenolink}[1]{\emph{#1}}

\begin{document}
\maketitle[width=0.92\textwidth]
% \maketitle
% \addlogo[north west]{(2,-1)}{9cm}{images/Grass_GIS}
%Please insert your institution logo here
% \addlogo[north east]{(-2,-2.5)}{4cm}{images/logo_FEM_CRI}
% \addlogo[north east]{(-2,-5.5)}{4cm}{images/NC_State_Seal}
% \addlogo[north east]{(-8,-2.5)}{4cm}{images/Logo_cvut}
% \addlogo[north east]{(-8,-6.5)}{4cm}{images/IWMI_logo}
% \addlogo[north east]{(-2,-10.5)}{4cm}{images/logo_ec-jrc}

\begin{columns}

%%%%%%%%%%%%%%%%%%%%%%%%%%%%%%%%%%%%%%%%%%%%%%%%%%%%%%%%%%%%%%%%%%%%%
%%%%%%%%%%%%%%%%%%%%%%%%%%%%%%%%%%%%%%%%%%%%%%%%%%%%%%%%%%%%%%%%%%%%%
%%%%%%%%%%%%%%%%%%%%%%%%%%%%%%%%%%%%%%%%%%%%%%%%%%%%%%%%%%%%%%%%%%%%%
%%%%%%%%%%%%%%%%%%%%%%%%%%%%%%%%%%%%%%%%%%%%%%%%%%%%%%%%%%%%%%%%%%%%%
\column{0.25}

%%%%%%%%%%%%%%%%%%%%%%%%%%%%%%%%%%%%%%%%%%%%%%%%%%%%%%%%%%%%%%%%%%%%%%%%%%%%%%%%
\block{\blocktitlewrap{Highlights}}
{
% \setlength{\parskip}{0.3ex}

\renewcommand{\labelitemi}{\textcolor{gray}{$\bullet$}\hspace{0.5ex}}
\newcommand{\blocksectiontitle}[1]{\bigskip\textbf{\textcolor{gray}{\textsf{#1}}}}

\blocksectiontitle{Poster topic highlights}

\begin{itemize}
 \item Algorithms and models included in GRASS GIS remain available long term \citep{chemin2015grass}.
 \item Analytical tools are not limited to one domain but spread across many fields.
 \item New tools can be built based on functionality or code of the existing ones
       regardless of the particular domain of problems they belong to.
 \item Both the functionality and the code are evaluated
       by the community of users and developers in different fields and scales.
% continuous automated tests (Petras, 2014 \cite{Petras2014}),
\end{itemize}

\blocksectiontitle{General GRASS GIS highlights}

\begin{itemize}
%  \item Community supports users and new developments through online
%  \item long-term moving forward
 \item The GRASS GIS development team takes care of interface and operating system related changes
       in the code provided by scientists and contributors.
 \item Cost estimated by Black Duck Open Hub is over \$30,000,000.
 % https://www.openhub.net/p/grass_gis/estimated_cost
 \item The free, libre and open source license used in GRASS GIS
       not only allows users to reduce GIS software costs to minimum,
       but also enables organizations to deploy cloud-based solutions without any license restrictions.
 \item GRASS GIS is used both directly and as a backend in other projects
       such as QGIS and R.
\end{itemize}

% oholoh hours

}


%%%%%%%%%%%%%%%%%%%%%%%%%%%%%%%%%%%%%%%%%%%%%%%%%%%%%%%%%%%%%%%%%%%%%
\block{\blocktitlewrap{Lidar Data Processing}}{

Filtering of ground and non-ground points was included into GRASS GIS
early on in the lidar era in the \gmodule{v.lidar.edgedetection} group of modules.
% TODO: ref
% v.lidar.correction Corrects the v.lidar.growing output. It is the last of the three algorithms for LIDAR .
% v.lidar.edgedetection Detects the object's edges from a LIDAR data set.
% v.lidar.growing Building contour determination and Region Growing algorithm for determining the building inside
The module \gmodule{v.surf.rst} for spatial interpolation was developed approximately 20 years
ago, but since then it has been improved several times \citep{tracvsurfrst}
including recent parallelization which will be included in GRASS GIS 7.4.
Currently the module is used for both digital terrain model interpolation and interpolations in general.

The module \gmodule{r.in.lidar} statistically analyzes massive point clouds.
% TODO: base_raster, ref
For advanced general point analysis GRASS GIS
\gmodule{v.outlier} implemented originally specifically for lidar data
and recently also \gmodule{v.cluster}
implementing variety of clustering methods including
DBSCAN (Density-Based Spatial Clustering of Applications with Noise)
and OPTICS (Ordering Points to Identify the Clustering Structure).
The \gmodule{v.outlier} module serves as a base for a user contributed module \gamodule{v.lidar.mcc}
implementing Multiscale Curvature Classification (MCC).
% TODO: ref

% v.delaunay v.voronoi New option to create Voronoi diagrams for areas.

\centering
\begin{minipage}{0.48\linewidth}
\centering
\includegraphics[width=0.7\textwidth]{elevation_lidar}
\end{minipage}
~
\begin{minipage}{0.48\linewidth}
\centering
\includegraphics[width=\textwidth]{lidar_profile}
\end{minipage}

\bigskip

\begin{minipage}{0.48\linewidth}
\centering
Digital elevation model interpolated from lidar point clouds
using \gmodule{v.surf.rst}. Data are showing tillage in an agricultural field near Raleigh (North Carolina, USA)
\end{minipage}
~
\begin{minipage}{0.48\linewidth}
\centering
Profile (vertical slice) of a small portion of a point cloud showing tree structure as captured by the returns
(\gamodule{v.profile.points})
\end{minipage}

}


%%%%%%%%%%%%%%%%%%%%%%%%%%%%%%%%%%%%%%%%%%%%%%%%%%%%%%%%%%%%%%%%%%%%%%%%%%%%%%%%
\block{\blocktitlewrap{Acknowledgements}}{

\newcommand{\listhspace}{\hspace{0.005\linewidth}}
\newcommand{\listlogowidth}{0.10\linewidth}
\newcommand{\listtextwidth}{0.82\linewidth}

\begin{minipage}{\listlogowidth}
\centering
\includegraphics[width=0.5\linewidth]{grass}
\end{minipage}
\listhspace
\begin{minipage}{\listtextwidth}
We acknowledge all members of the GRASS GIS community both users and developers.
We would like to thank Markus Metz, Anna Petrasova, Stepan Turek, and Radim Blazek.
\end{minipage}

\bigskip

\begin{minipage}{\listlogowidth}
\includegraphics[width=\linewidth]{osgeo}
\end{minipage}
\listhspace
\begin{minipage}{\listtextwidth}
Open Source Geospatial Foundation (OSGeo)
supports the collaborative development of open source geospatial software.
GRASS GIS is a OSGeo project and
OSGeo provides infrastructure for project
websites, mailing lists and source code management.
\end{minipage}

\bigskip

\begin{minipage}{\listlogowidth}
\includegraphics[width=\linewidth]{google}
\end{minipage}
\listhspace
\begin{minipage}{\listtextwidth}
Initial development of \gmodule{i.segment} module as well as several other developments
were done as a part of the Google Summer of Code project.
Google provides financial support to students and organizations participating in the Google Summer of Code project.
\end{minipage}

\vspace{0.2cm}

\textcolor{gray}{
\hrulefill
}

\vspace{0.1cm}

\newcommand{\qrcodesize}{0.05\linewidth}

% qrencode http://grass.osgeo.org -o qr_grass.eps -t EPS
% epspdf -b qr_grass.eps qr_grass.pdf

\begin{center}
\begin{tabular}{c}

% \hspace{5mm}

\begin{minipage}{\qrcodesize}
\includegraphics[width=\textwidth]{./images/qr_grass.pdf}
\end{minipage}
~
\begin{minipage}{0.15\linewidth}
\small {\href{http://grass.osgeo.org}{\nolinkurl{grass.osgeo.org}}}
\end{minipage}

\begin{minipage}{0.1\linewidth}
\href{http://creativecommons.org/licenses/by-sa/4.0/}{\includegraphics[width=\textwidth]{ccbysa}}
\end{minipage}
~
\begin{minipage}{0.35\linewidth}
\small This poster is licensed under a Creative Commons Attribution-ShareAlike 4.0 International License.
\end{minipage}

\end{tabular}
\end{center}

\vspace{-0.08cm}
}


%%%%%%%%%%%%%%%%%%%%%%%%%%%%%%%%%%%%%%%%%%%%%%%%%%%%%%%%%%%%%%%%%%%%%
%%%%%%%%%%%%%%%%%%%%%%%%%%%%%%%%%%%%%%%%%%%%%%%%%%%%%%%%%%%%%%%%%%%%%
%%%%%%%%%%%%%%%%%%%%%%%%%%%%%%%%%%%%%%%%%%%%%%%%%%%%%%%%%%%%%%%%%%%%%
%%%%%%%%%%%%%%%%%%%%%%%%%%%%%%%%%%%%%%%%%%%%%%%%%%%%%%%%%%%%%%%%%%%%%
\column{0.25}


%%%%%%%%%%%%%%%%%%%%%%%%%%%%%%%%%%%%%%%%%%%%%%%%%%%%%%%%%%%%%%%%%%%%%
%%%%%%%%%%%%%%%%%%%%%%%%%%%%%%%%%%%%%%%%%%%%%%%%%%%%%%%%%%%%%%%%%%%%%%%%%%%%%%%%%
\block{\blocktitlewrap{Water, Floods and Erosion}}{

GRASS GIS entails several modules that constitute the result of active research on natural hazards.
The \gmodule{r.sim.water} simulation model \citep{Mitas1998b}
for overland flow with spatially variable rainfall excess conditions was integrated into the Emergency
Routing Decision Planning system as a WPS \citep{raghavan2014deploying}.
The module \gmodule{r.sim.water} together with
the module \gmodule{r.sim.sediment} for erosion-deposition modeling
implements a path sampling algorithm which is robust and easy to parallelize.
A unique least cost path algorithm which doesn't need any depression filling
was implemented in \gmodule{r.watershed} module in 1989
and it was updated for massive data sets in 2011.
% TODO: refs
The \gmodule{r.sim.water} module was also utilized by \cite{Petrasova2014} and is now part of
\emph{Tangible Landscape}, a tangible GIS system, which also incorporated \gmodule{r.damflood},
a dam break inundation simulation by \cite{cannata2012two}.

\bigskip

\vspace*{1cm}

\centering

\begin{minipage}{0.49\linewidth}
\centering
\includegraphics[width=0.7\textwidth]{rsimwater_architects}
\end{minipage}
~
\begin{minipage}{0.49\linewidth}
\includegraphics[width=\textwidth]{damflood_tangible}
\end{minipage}

\bigskip

\begin{minipage}{0.49\linewidth}
\centering
Overland flow simulated by \gmodule{r.sim.water} used for landscape
architecture design in Tangible Landscape environment
(Historical master plan for Lake Raleigh, NC, USA)
\end{minipage}
~
\begin{minipage}{0.49\linewidth}
Dam breach on Lake Raleigh (NC, USA) in Tangible Landscape environment simulated using \gamodule{r.damflood} module
\end{minipage}


% \vspace*{1.4cm}
}



%%%%%%%%%%%%%%%%%%%%%%%%%%%%%%%%%%%%%%%%%%%%%%%%%%%%%%%%%%%%%%%%%%%%%%%%%%%%%%%
\block{\blocktitlewrap{Image Segmentations}}{
\gamodule{r.smooth.seg} (formerly \gmodulenolink{r.seg})
was created by Alfonso Vitti \citep{vitti2008free, vitti2012mumford}
and implemented a piece-wise smooth approximation of the original image
according to \cite{mumford1989optimal} and \cite{march1997variational}
which can be used to reduce noise in the original image.

This supplemented \gmodule{r.clump} available from 1980s
which groups pixels with the same categories (or integer values).
The latest version of \gmodule{r.clump} coming in GRASS GIS 7.4
supports multiple image bands (or any rasters maps) as input.
Clumping of cells based on threshold value and clumping
of double precision floating point input is supported as well in the new version.

Eric Momsen implemented initial version of region-growing image segmentation in 2012
and Markus Metz then extended and optimized the code resulting in
the inclusion of the \gmodule{i.segment} module into GRASS GIS 7.0
and final replacement of old multi-resolution classification module of the same name
which was available in 1992 \citep{zhuang1992image}.
Another module called \gamodule{i.segment.hierarchical} by Pietro Zambelli is based on \gmodule{i.segment}
and performs parallelized hierarchical segmentation.

\vspace*{0.7cm}

\begin{minipage}{0.5\linewidth}
\begin{center}
\includegraphics[width=\textwidth]{superpixels_slic_pseudo}
\end{center}
\end{minipage}
~
\begin{minipage}{0.5\linewidth}
\begin{center}
\includegraphics[width=\textwidth]{superpixels_slic_colored}
\end{center}
\end{minipage}
\vspace{2mm}
\begin{center}
Superpixels (black outlines) on pseudo-color image of central Wake county, NC, USA (left)
and the same superpixels colored according to the mean NDVI value per pixel (right).
\end{center}

\vspace*{1cm}

In 2016 GRASS GIS implementation of SLIC Superpixels segmentation \citep{achanta2010epfl, achanta2012slic}
was requested by the community
and Rashad Kanavath and Markus Metz implemented \gmodule{i.superpixels.slic}
which provides users both with SLIC and SLIC0 methods.


}


%%%%%%%%%%%%%%%%%%%%%%%%%%%%%%%%%%%%%%%%%%%%%%%%%%%%%%%%%%%%%%%%%%%%
%%%%%%%%%%%%%%%%%%%%%%%%%%%%%%%%%%%%%%%%%%%%%%%%%%%%%%%%%%%%%%%%%%%%%
\column{0.25}

%%%%%%%%%%%%%%%%%%%%%%%%%%%%%%%%%%%%%%%%%%%%%%%%%%%%%%%%%%%%%%%%%%%%%%%%%%%%%%%%
\block{\blocktitlewrap{Topology, Cleaning, Overlays, Attributes}}{
Besides basic vector analysis tools such as \gmodule{v.buffer}
and \gmodule{v.overlay},
suite of functionality for topology checking and cleaning is available
through \gmodule{v.build} and \gmodule{v.clean} modules.
The vector cleaning tools are particularly advantageous considering the use of non-topological exchange formats.
The \gmodule{v.clean} module was introduced in 2002 and several different contributors
extended its functionality with last large set of improvements included in GRASS GIS 7.0.
For that release all topological cleaning tools have been optimized
with regard to processing speed, robustness, and system requirements.
The processing speed of the \gmodule{v.clean} module was substantially improved as well.

\vspace*{1cm}

\begin{center}
\begin{minipage}{0.9\linewidth}
\begin{center}
\includegraphics[width=.3\textwidth]{topo_original}
~
\includegraphics[width=.3\textwidth]{topo_errors}
~
\includegraphics[width=.3\textwidth]{topo_fixed}
Original vector imported without cleaning (left),
identified errors (middle),
automatically topologically corrected vector (right)
\end{center}
\end{minipage}
\end{center}

% v.in.ascii input= output=imported format=standard
% v.build -e map=imported error=build_errors
% v.clean -c input=imported output=clean error=cleaning_errors tool=snap,rmdangle,rmbridge,chbridge,bpol,prune threshold=5
% d.vect map=build_errors color=255:33:36 fill_color=none width=5 icon=basic/point size=30
% d.vect map=cleaning_errors color=255:33:36 fill_color=none width=5 icon=basic/point size=30
% d.vect map=imported color=56:16:108 fill_color=205:64:113 width=5

\vspace*{1cm}

An experimental module \gamodule{v.feature.algebra} (formerly \gmodulenolink{v.mapcalc}) from 2002
% actually, the original by Radim Blazek (Jun 11, 2002) seems to be more similar to the current v.mapcalc
% it was using Vect_overlay_str_to_operator Vect_overlay
% it was replaced by code by Christoph Simon in Aug 7, 2002
was replaced by Python for GRASS GIS C libraries (PyGRASS) in GRASS GIS 7.0.
However, new \gamodule{v.mapcalc} module is available with richer syntax and functionality.

Attribute processing and queries in GRASS GIS can take an advantage of latest developments
in database management systems (DBMSs) as several DBMSs are supported most notably SQLite and PostgreSQL.
A PostGIS database can be connected including its geometry and topology.
In parallel to that native support the OGC Simple Features is available.

The GRASS GIS 7 releases come with faster vector processing backend
and more efficient internal vector format;
both disk and memory requirements were reduced.
The new spatial index performs queries faster (more than 10 times for large vectors compared to GRASS GIS 6).

}


%%%%%%%%%%%%%%%%%%%%%%%%%%%%%%%%%%%%%%%%%%%%%%%%%%%%%%%%%%%%%%%%%%%%%
\block{\blocktitlewrap{Spatio-Temporal Data Analysis}}{
The time dimension was introduced in GRASS GIS 7.0 for vector maps, rasters, and 3D rasters
which transformed GRASS GIS into a fully-featured temporal GIS \citep{Gebbert20141, gebbert2015grass}.
Time series of map layers are managed in space time datasets, a new data type in GRASS GIS,
and are still accessible also as individual map layers.
Based on the GRASS GIS Temporal Framework Python application programming interface (API),
more than 50 modules were implemented to manage, analyze, process
and visualize space time datasets.
More than 100,000 map layers can be now handled efficiently in GRASS GIS.
%
Example usages of this functionality include
analysis of the
European Climate Assessment \& Dataset ECA\&D \citep{Haylock2008_climate_series}
and temperate climate zone in the European Union identification \citep{Gebbert20141}.

The new temporal modules (staring with \gmodulenolink{t.})
work beside well established \gmodule{r.series} module
and specialized modules such as \gamodule{r.hants} implemented according to \cite{roerink2000reconstructing}
or \gamodule{r.seasons}.
Latest addition includes raster and vector temporal algebra
which can be used for tasks such as computing hydrothermal coefficient for a time series of climate data
using the actual mathematical formula \citep{leppelt2015grass}.

\vspace*{1.5cm}

\begin{minipage}{\linewidth}
\centering
\includegraphics[width=.7\linewidth]{images/temporal_precip_temp}
\\
Creating a synchronized animation of monthly total precipitation and mean temperature for NC, USA
\end{minipage}

\vspace*{1cm}

}



%%%%%%%%%%%%%%%%%%%%%%%%%%%%%%%%%%%%%%%%%%%%%%%%%%%%%%%%%%%%%%%%%%%%%
%%%%%%%%%%%%%%%%%%%%%%%%%%%%%%%%%%%%%%%%%%%%%%%%%%%%%%%%%%%%%%%%%%%%%
%%%%%%%%%%%%%%%%%%%%%%%%%%%%%%%%%%%%%%%%%%%%%%%%%%%%%%%%%%%%%%%%%%%%%
%%%%%%%%%%%%%%%%%%%%%%%%%%%%%%%%%%%%%%%%%%%%%%%%%%%%%%%%%%%%%%%%%%%%%
\column{0.25}


%%%%%%%%%%%%%%%%%%%%%%%%%%%%%%%%%%%%%%%%%%%%%%%%%%%%%%%%%%%%%%%%%%%%%%%%%%%%%%%%
\block{\blocktitlewrap{Vector Network Analysis}}{
In 2003 Radim Blazek introduced shortest path analysis (\gmodule{v.net.path}),
traveling salesman (\gmodule{v.net.salesman}),
and several other modules for vector network analysis (staring with \gmodule{v.net}).
Over the years, the number of available algorithms increased to over 15 modules
including, for example, centrality measures and connected components added in 2009 by Daniel Bundala.
In 2014 Stepan Turek implemented turns support into all relevant vector network modules.
Also since GRASS GIS 7.0 all vector network analysis tools provide fine control over node costs.
% combination with r.cost/r.walk workflows \citep{Petrasova2014}

\bigskip

\centering
\begin{minipage}{0.5\linewidth}
\begin{center}
\includegraphics[width=\textwidth]{network}
Shortest path between two points in the Wake County road network (NC, USA)
\end{center}
\end{minipage}

}


\block{\blocktitlewrap{Landscape Structure}}{
A set of modules for multiscale analysis of landscape structure was added in 1992
by \cite{baker1992r}, who developed the \gmodulenolink{r.le} model similar to
FRAGSTATS \citep{mcgarigal1995fragstats}.
The modules were gradually improved to become \gmodule{r.li} in 2006.
Further development continued, with a significant
increase in speed \citep{tracrli} and a new interactive user interface.
\cite{rocchini2013calculating} used \gmodule{r.li} modules to implement
high level tool for calculating landscape diversity.

\vspace*{0.5cm}

\begin{minipage}{0.5\linewidth}
\includegraphics[width=\textwidth, trim={0 180 0 0}, clip]{diversity_classes}
\end{minipage}
\begin{minipage}{0.5\linewidth}
\includegraphics[width=\textwidth, trim={0 180 0 0}, clip]{diversity_shannon}
\end{minipage}

\vspace*{1ex}

Landuse classes and derived landscape diversity according to Shannon index near Charlotte, NC, USA
% r.diversity input=development_2006 prefix=diversity alpha=0.5 size=65
% r.li.shannon input="development_2006" config="conf_diversity_65.0" output="diversity_shannon_size_65.0"
}



%%%%%%%%%%%%%%%%%%%%%%%%%%%%%%%%%%%%%%%%%%%%%%%%%%%%%%%%%%%%%%%%%%%%%%%%%%%%%%%%
\block{\blocktitlewrap{References}}{

\vspace{-0.2cm}
\scriptsize

% \newcommand{\blocksectiontitle}[1]{\subsubsection*{\textcolor{gray}{\textsf{#1}}}}
\newcommand{\blocksectiontitle}[1]{\textbf{#1}}

%\blocksectiontitle{References}
\begingroup
\renewcommand{\section}[2]{}%
\bibliographystyle{apalike}
\bibliography{poster}
\endgroup

}



\end{columns}

\end{document}
